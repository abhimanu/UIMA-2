\documentclass{article}
\usepackage[utf8]{inputenc}

\title{Hw2 Documentation, LTI 11791}
\author{Abhimanu Kumar \\ Andre Id - ABHIMANK}
\date
%\usepackage{natbib}
%\usepackage{graphicx}

\begin{document}

\maketitle

\section{Architectutre Design}
I describe here the design considerations, UIMA issues and other 
NLP issues that I took into account while designing the logical architecture 
and implementing UIMA analysis engine design.
We have a total of 5 annotator classes: 1) TokenAnnotator, 2)NGrammAnnotator,
3) QuestionAnnotator, 4)QuestionAnnotator, and finally 5)AnswerScoreAnnotator.
Each class has a corresponding Analysis Engine Descriptor XML file that
defines its input output set and types used.

\subsection{Class TokenAnnotator}
This class is the basic token annotator class that annotates every token 
encountered in the dataset. It stores the basic tokens encountered in the text. 

\subsection{Class NgramAnnotator}
This is the class that encapsulates the group of tokens that are taken as 
bi-gram tri-gram etc. I create uni-grams, bi-grams as well as tri-grams, and 
all that creation logic in the same class. It creats a set of overlapping
bi-grams and tri-grams.


\subsection{Class QuestionAnnotator}
This class encapsulates the methd to get the sentence body of the question. 
This class just extracts the begin and end of the question.
 
\subsection{Class AnswerAnnotator}
This class encapsulates the answers provided to a question. Apart from
extracting the end and begin of the answer it also stores the correctness of the
answer.

\section{Special Points}
1) I have coded logic to create uni-gram, bi-gram and tri-grams in a single
class. This helps in code reusability as well as compactness of the code.

2) The scoring function used is the number of NGram overlap between an answer
and its question. It is a simple yet good scoring method. It gives a precision
of 0.5 nnd 0.6666 on the two documents given as input.

3) The precision is defined as (number of correct answers in top N)/N where N is
the total number of correct answer.

\section{Checking The Output}
Run the hw2-abhimank-aae.xml, the aggregate analysis engine, and it gives the 
precision of the documents in the console.  
%\bibliographystyle{plain}
%\bibliography{references}
\end{document}
